\documentclass[english,12pt]{article}
\usepackage[]{geometry}
\usepackage{tabularx}
\usepackage{blindtext}
\usepackage{enumerate}
\usepackage{parskip} 
\usepackage{siunitx}
\usepackage{amsmath}
\usepackage{amsfonts}
\usepackage{amssymb}
\usepackage{hyperref}
\usepackage{listings}
\usepackage{inconsolata}
\usepackage{parskip}
\usepackage{graphicx}
\usepackage{wrapfig}
\usepackage{float}
\usepackage{titlesec}
\usepackage{blindtext}
\usepackage{longtable}
\usepackage{multicol}
\usepackage{varwidth}
\usepackage{multirow}
\usepackage{booktabs}
\usepackage[svgnames,table]{xcolor}
\usepackage[tableposition=above]{caption}
\usepackage{pifont}
\usepackage{array,ragged2e}
\newcolumntype{P}[1]{>{\RaggedRight\arraybackslash}p{#1}}
\titleformat{\section}{\large\bfseries}{\thesection}{1em}{}
\hfuzz=0.64pt % Allow hbox to overflow slightly

\begin{document}
\begin{titlepage}
    \null
    \vspace*{2cm}
    
    \begin{center}
        {\Huge \bfseries CSE 363: Computer Security}\\[1.5cm]
        {\Large \bfseries Final Project Proposal}\\[2cm]
        
        \textbf{Authors:} \\[0.5cm]
        Cody Johnston - \textit{cody.johnston@student.nmt.edu}\\
        Cole Johnson - \textit{cole.johnson@student.nmt.edu}\\
        Colin Grandjean - \textit{colin.grandjean@student.nmt.edu}\\
        John Runyon - \textit{john.runyon@student.nmt.edu}\\
        
        New Mexico Institute of Mining and Technology\\
        Socorro, NM 87801, USA\\[2cm]
        
        {\large Date: March 3, 2025}
    \end{center}
    
    \vfill
    \hrule
    \smallskip
    \centerline{\sc New Mexico Tech}
    \smallskip
    \hrule
\end{titlepage}
\frenchspacing
\pagebreak
\section*{Introduction and Outline of Project}
\subsection*{The Basics of Botnets}
A botnet (the combination of the words robot and network) is a group of network 
enabled malware infected. Bots, the infected devices, 
are controlled by a central server or servers, 
often called the “bot-herder” or the “command and control server”. 
\subsection*{Outline of Project}
Here is what we would tentatively like to cover in each section of our paper:
\begin{enumerate}[\bf (a.)]
    \item Introduction
    \begin{enumerate}[(a.)]
        \item What are botnets?
        \item Examples of botnets
        \item Famous cases of botnet attacks
    \end{enumerate}
    \item Taxonomy of Botnets
    \begin{enumerate}[\bf (a.)]
        \item Architecture / Design
        \item Communication Methods
        \item Purpose
    \end{enumerate}
    \item Attack Vectors
    \begin{enumerate}[(a.)]
        \item Vulnerabilities
        \item Phishing
        \item Worm-Like
        \item External Devices
    \end{enumerate}
    \item Mitigation Techniques
    \begin{enumerate}[(a.)]
        \item Firewalls
        \item Intrusion Detection Systems (IDS)
        \item Challenge and Response Tests
    \end{enumerate}
    \item Experimentation
    \begin{enumerate}
        \item Existing botnet frameworks
        \item Developing a basic botnets
        \item Implementing mitigation techniques (if time permits)
    \end{enumerate}
\end{enumerate}
\subsection*{Taxonomy of Botnets}
Botnets can be constructed in a number of 
ways and fulfill many different purposes for an attacker. 
They can either be centralized, using a client-server model, or decentralized, 
using a peer-to-peer model. 
\subsection*{Attack Vectors}
Botnets exploit various attack vectors to infect devices 
and expand their network. The most common methods that botnets 
use are exploiting vulnerabilities, phishing, worm-like propagation, 
and using external devices. Botnets exploit vulnerabilities such as 
unpatched software, misconfigurations, and security loopholes 
found within operating systems, applications, or IoT devices. 
\subsection*{Mitigation Techniques}
There any a myriad of different techniques that try to mitigate the damage
caused by botnets, or limit their spread altogether. We would like to focus
on a few different techniques: firewalls, intrusion detection systems, and
challenge response tests. 
\subsection*{Experimentation}
We first plan to look into existing 
botnet frameworks to examine their methods and features. 
After research into frameworks, we plan to build a simulated botnet, 
with various bot devices running in containers. 
Using a modularized approach we can change the purpose of the botnet 
dynamically to explore the differences. We plan to compare botnet 
communication and architecture methods by running a set of tasks for the 
\end{document}
